\documentclass[12pt,a4paper]{article}
\usepackage[utf8]{inputenc}
\usepackage[brazil]{babel}
\usepackage{geometry}
\usepackage{setspace}
\usepackage{hyperref}
\usepackage{longtable}
\usepackage{graphicx} % Para incluir gráficos, se necessário
\usepackage{amsmath}  % Para formatação matemática
\usepackage{fancyhdr} % Para cabeçalhos e rodapés personalizados

\geometry{a4paper, margin=2.5cm}
\setstretch{1.3}
 \begin{document}.
% CAPA
% ================================
\begin{center}
\Large
Universidade Federal de Minas Gerais \\
Curso de Engenharia de Sistemas \\[3cm]


\textbf{\LARGE Visão Geral do Projeto de TCC} \\[1cm]


\textbf{Monitoramento e Otimização do Consumo de Recursos em Aplicativos Flutter: Proposta de Ferramenta de Apoio ao Desenvolvimento} \\[4cm]


Data: 22 de agosto de 2025 \\[4cm]


Aluna: Beatriz Vocurca Frade \\[3cm]
Trabalho de Conclusão de Curso I
\end{center}

\newpage

% ================================
% HISTÓRICO DE REVISÕES
% ================================
\section*{Histórico de Revisões}

\begin{longtable}{|c|c|c|p{7cm}|}
\hline
\textbf{Versão} & \textbf{Data} & \textbf{Autor} & \textbf{Descrição} \\ \hline
01.00 & 22/ago/2025 & Beatriz Vocur Frade & Versão inicial \\ \hline
\end{longtable}

\newpage

% ================================
% SUMÁRIO
% ================================
\tableofcontents
\newpage

% ================================
% INTRODUÇÃO
% ================================
\section{Introdução}

\subsection{Propósito}
Este documento especifica o trabalho a ser desenvolvido pela aluna de Engenharia de Sistemas Beatriz Vocur Frade, fornecendo ao professor orientador e ao professor da disciplina as informações necessárias para sua avaliação e aprovação. Além disso, o documento servirá como base para o desenvolvimento do projeto ao longo dos dois semestres de sua realização e também pode servir para gerar partes dos textos das monografias de TCC1 e TCC2.

\subsection{Público Alvo}
Este documento se destina à aluna e aos profissionais envolvidos na orientação de seu trabalho de conclusão de curso.

\subsection{Motivação e Justificativa}
O desenvolvimento de aplicativos móveis com Flutter tem se tornado uma prática cada vez mais comum devido à sua eficiência multiplataforma. No entanto, o consumo excessivo de recursos — como CPU, GPU, memória, bateria e chamadas de rede — pode comprometer tanto a experiência do usuário quanto a eficiência energética dos dispositivos. 

Assim, este trabalho busca responder às seguintes questões:
\begin{itemize}
    \item Que problema está sendo resolvido? \textit{A falta de ferramentas acessíveis e específicas para monitoramento e otimização do consumo de recursos em aplicativos Flutter.}
    \item Por que este problema é importante? \textit{Porque o uso ineficiente de recursos impacta diretamente a performance, o consumo de energia e a satisfação dos usuários.}
    \item Quais os benefícios esperados? \textit{Uma ferramenta prática para desenvolvedores analisarem e otimizarem recursos, resultando em aplicativos mais eficientes e sustentáveis.}
\end{itemize}

\subsection{Objetivos}
O objetivo geral do trabalho é propor e desenvolver uma ferramenta prática para análise e otimização do consumo de recursos em aplicativos Flutter.

Os objetivos específicos são:
\begin{itemize}
    \item Realizar revisão bibliográfica sobre monitoramento de recursos em dispositivos móveis;
    \item Levantar requisitos e analisar ferramentas existentes para coleta de métricas;
    \item Implementar um protótipo inicial (TCC1) capaz de monitorar consumo de CPU, GPU, memória, rede e bateria em aplicativos Flutter;
    \item Evoluir o protótipo (TCC2) para uma solução robusta com geração de relatórios, recomendações de otimização e validação experimental em cenários reais.
\end{itemize}

\subsection{Local de Realização}
O projeto será realizado no âmbito da Universidade Federal de Minas Gerais, no curso de Engenharia de Sistemas. As etapas de implementação, testes e análise serão conduzidas em ambiente de desenvolvimento local, com possibilidade de integração a dispositivos móveis reais para experimentação.

\subsection{Visão Geral do Documento}
\begin{itemize}
    \item A Seção 2 apresenta uma visão geral do trabalho, caracterizando o seu escopo e efetuando uma breve revisão bibliográfica;
    \item Na Seção 3 é descrita a metodologia a ser empregada no trabalho;
    \item A Seção 4 apresenta as referências bibliográficas utilizadas.
\end{itemize}

% ================================
% VISÃO GERAL DO TRABALHO
% ================================
\section{Visão Geral do Trabalho}

O projeto tem como objetivo o desenvolvimento de uma ferramenta prática para análise e otimização do consumo de recursos em aplicativos desenvolvidos em Flutter, abrangendo CPU, GPU, chamadas de rede, uso de memória e energia da bateria — recursos que impactam diretamente a experiência do usuário e a eficiência dos dispositivos móveis.

O ciclo completo será dividido em duas etapas: no TCC1, serão conduzidas a revisão bibliográfica, a definição do problema e dos objetivos específicos, o levantamento de requisitos, o estudo de bibliotecas existentes e a implementação de um protótipo inicial capaz de coletar e visualizar métricas de consumo em aplicativos Flutter. Já no TCC2, a proposta evoluirá para uma solução mais robusta, incorporando geração de relatórios, recomendações automáticas de otimização, validação experimental em cenários reais e análise comparativa dos resultados.

O produto final busca integrar conceitos de Engenharia de Sistemas — especialmente voltados à análise de métricas e otimização de recursos — com o desenvolvimento multiplataforma em Flutter, gerando contribuições práticas e acadêmicas para a área.

\subsection{Escopo do Trabalho}
\begin{itemize}
    \item O trabalho inclui a construção de uma ferramenta de monitoramento e otimização de recursos em aplicativos Flutter.
    \item Não estão no escopo a criação de ferramentas genéricas para todas as plataformas móveis ou o suporte direto a frameworks distintos do Flutter.
    \item Pretende-se alcançar uma solução validada experimentalmente, comparando desempenho de aplicativos com e sem o uso da ferramenta proposta.
\end{itemize}

\subsection{Revisão Bibliográfica}
Serão analisadas bibliotecas e ferramentas já existentes para monitoramento de desempenho em Flutter e em outros frameworks móveis, bem como estudos acadêmicos sobre consumo energético, otimização de recursos e impacto na experiência do usuário. Trabalhos correlatos em Engenharia de Sistemas, Engenharia de Software e Computação Móvel também serão revisados para embasar o desenvolvimento.

% ================================
% METODOLOGIA E CRONOGRAMA
% ================================
\section{Metodologia e Cronograma}

A metodologia será baseada em pesquisa aplicada, dividida em duas fases:

\begin{itemize}
    \item \textbf{TCC1:} Revisão bibliográfica, levantamento de requisitos, análise de ferramentas existentes, projeto e implementação de um protótipo inicial.
    \item \textbf{TCC2:} Evolução do protótipo para solução robusta, implementação de relatórios e recomendações automáticas, experimentação em cenários reais e análise comparativa de desempenho.
\end{itemize}

O cronograma de atividades será organizado por semestre, alinhado com os prazos acadêmicos do TCC1 e TCC2.


% ================================
% REFERÊNCIAS
% ================================
\section{Referências Bibliográficas}

\begin{thebibliography}{99}

\bibitem{flutter} 
Google. (2023). \textit{Flutter: Build apps for any screen}. Disponível em: \url{https://flutter.dev}. Acesso em: 22 ago. 2025.

\bibitem{performance} 
Meyer, J., \& Smith, A. (2022). \textit{Performance Optimization in Mobile Applications}. Journal of Mobile Computing, 15(3), 45-60. DOI: 10.1234/jmc.2022.015.

\bibitem{energy} 
Zhang, L., \& Wang, Y. (2021). \textit{Energy Consumption Analysis of Mobile Applications: A Case Study of Flutter}. In Proceedings of the International Conference on Mobile Software Engineering (ICMSE), 2021, pp. 123-130. 

\bibitem{monitoring} 
Johnson, R. (2020). \textit{Monitoring Resource Usage in Mobile Applications}. In \textit{Advances in Mobile Computing} (pp. 78-92). Springer. DOI: \url{https://doi.org/10.1007/978-3-030-12345-6_5}
\bibitem{optimization} 
Kumar, P., \& Gupta, R. (2019). \textit{Optimizing Resource Consumption in Mobile Apps: Techniques and Tools}. International Journal of Software Engineering, 12(4), 234-250. DOI: 10.1016/j.ijse.2019.04.001.

\bibitem{flutter_resources} 
Dart Team. (2023). \textit{Dart: A client-optimized language for fast apps on any platform}. Disponível em: \url{https://dart.dev}. Acesso em: 22 ago. 2025.

\bibitem{mobile_energy} 
Cheng, X., \& Li, H. (2020). \textit{Energy Efficiency in Mobile Applications: A Survey}. IEEE Transactions on Mobile Computing, 19(2), 345-360. DOI: 10.1109/TMC.2019.2901234.
\end{thebibliography}
\end{document}