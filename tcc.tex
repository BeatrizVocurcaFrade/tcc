\documentclass[12pt,a4paper]{article}
\usepackage[utf8]{inputenc}
\usepackage[brazil]{babel}
\usepackage{geometry}
\usepackage{setspace}
\usepackage{hyperref}
\usepackage{longtable}
\usepackage{graphicx}
\usepackage{amsmath}
\usepackage{fancyhdr}
\usepackage{array}

\geometry{a4paper, margin=2.5cm}
\setstretch{1.3}

\begin{document}

% CAPA
% ================================
\begin{center}
\Large
Universidade Federal de Minas Gerais \\
Curso de Engenharia de Sistemas \\[3cm]

\textbf{\LARGE Visão Geral do Projeto de TCC} \\[1cm]

\textbf{Monitoramento e Otimização do Consumo de Recursos em Aplicativos Flutter: Proposta de Ferramenta de Apoio ao Desenvolvimento} \\[4cm]

Data: 22 de agosto de 2025 \\[4cm]

Aluna: Beatriz Vocurca Frade \\[3cm]

Trabalho de Conclusão de Curso I
\end{center}

\newpage

% ================================
% HISTÓRICO DE REVISÕES
% ================================
\section*{Histórico de Revisões}

\begin{longtable}{|c|c|c|p{7cm}|}
\hline
\textbf{Versão} & \textbf{Data} & \textbf{Autor} & \textbf{Descrição} \\ \hline
01.00 & 22/ago/2025 & Beatriz Vocurca Frade & Versão inicial \\ \hline
\end{longtable}

\newpage

% ================================
% SUMÁRIO
% ================================
\tableofcontents
\newpage

% ================================
% INTRODUÇÃO
% ================================
\section{Introdução}

\subsection{Propósito}
Este documento apresenta e especifica o trabalho de conclusão de curso a ser desenvolvido pela aluna Beatriz Vocurca Frade no âmbito do curso de Engenharia de Sistemas da Universidade Federal de Minas Gerais. Sua função é estabelecer uma base formal para o desenvolvimento da pesquisa, servindo como instrumento de avaliação inicial por parte do professor orientador e dos docentes da disciplina de TCC, além de constituir um guia estruturado que permitirá a expansão dos conteúdos aqui delineados em capítulos futuros das monografias correspondentes ao TCC1 e ao TCC2. Trata-se de um texto que busca não apenas explicitar objetivos e delimitações, mas também contextualizar a relevância acadêmica e prática do trabalho, permitindo que a proposta se mostre clara quanto à sua contribuição científica e tecnológica.

\subsection{Público-Alvo}
O público-alvo imediato deste documento é formado pela própria aluna responsável pelo desenvolvimento, pelo professor orientador e pelos docentes vinculados à disciplina de TCC, incluindo os avaliadores internos e externos responsáveis pela análise do projeto. Em um sentido ampliado, a pesquisa interessa também a pesquisadores de Engenharia de Software, profissionais do setor de desenvolvimento mobile, engenheiros de sistemas envolvidos em aplicações críticas de consumo energético e estudiosos das áreas de eficiência computacional e sustentabilidade digital. Assim, embora seja um documento orientado a um contexto acadêmico específico, seu conteúdo possui pertinência prática e pode dialogar com a comunidade técnica de desenvolvedores de software.

\subsection{Motivação e Justificativa}
O desenvolvimento e a utilização de aplicativos móveis têm crescido de maneira exponencial ao longo da última década, representando uma das principais vertentes da inovação tecnológica contemporânea. Nesse contexto, o Flutter consolidou-se como um dos frameworks mais utilizados, graças à sua capacidade de permitir o desenvolvimento multiplataforma com desempenho elevado e à ampla rede de suporte proveniente de sua comunidade ativa \cite{flutter}. Entretanto, a execução de aplicativos móveis envolve, de forma inevitável, o consumo de recursos computacionais fundamentais como CPU, GPU, memória, rede e bateria, elementos que influenciam diretamente a experiência do usuário final e impactam a longevidade e o desempenho dos dispositivos \cite{energy, monitoring}.  

Na literatura acadêmica, encontra-se ampla discussão acerca da necessidade de eficiência energética e de uso racional de recursos como parte integrante do desenvolvimento de software moderno \cite{mobile_energy, performance}. Essa preocupação transcende aspectos técnicos imediatos e conecta-se a questões de sustentabilidade digital, uma vez que a redução do consumo energético em escala global pode trazer impactos positivos na diminuição da pegada de carbono gerada por sistemas computacionais. Apesar da relevância do tema, o ecossistema Flutter ainda carece de soluções acessíveis e integradas que combinem o monitoramento detalhado do uso de recursos com recomendações práticas de otimização. Ferramentas disponíveis, como DevTools e Perfetto, embora poderosas, apresentam barreiras de adoção por serem excessivamente complexas e pouco amigáveis a desenvolvedores iniciantes ou de nível intermediário, o que limita sua aplicação em projetos cotidianos.  

Nesse cenário, o presente trabalho busca preencher a lacuna identificada ao propor o desenvolvimento de uma ferramenta que una o monitoramento de métricas de consumo de recursos a sugestões automáticas de otimização. A relevância da proposta encontra-se, portanto, em duas dimensões complementares. No âmbito acadêmico, o projeto contribui para a ampliação do corpo de conhecimento em Engenharia de Software aplicada a dispositivos móveis, incorporando reflexões sobre desempenho, sustentabilidade e qualidade de sistemas. No âmbito prático, a iniciativa pretende disponibilizar para a comunidade de desenvolvedores uma solução acessível, que democratize o acesso a práticas avançadas de análise de eficiência computacional.

\subsection{Objetivos}
O objetivo geral deste trabalho é propor e desenvolver uma ferramenta prática capaz de realizar a análise, o monitoramento e a otimização do consumo de recursos em aplicativos móveis desenvolvidos no framework Flutter, abarcando dimensões relacionadas ao processamento de CPU e GPU, ao uso de memória, às chamadas de rede e ao consumo energético da bateria, variáveis que impactam de maneira direta a experiência do usuário e a eficiência dos dispositivos.  

De forma articulada a esse objetivo maior, estabelecem-se metas intermediárias que orientam o percurso do trabalho. Pretende-se realizar uma revisão bibliográfica abrangente que fundamente teoricamente o problema e situe o estado da arte na área de monitoramento de desempenho e consumo em dispositivos móveis. Em seguida, será elaborado um levantamento sistemático dos requisitos funcionais e não funcionais da ferramenta proposta, garantindo clareza quanto às funcionalidades esperadas e aos critérios de desempenho e usabilidade. Outro objetivo específico é a análise crítica de ferramentas já existentes, tanto no ecossistema Flutter quanto em frameworks concorrentes como React Native, Kotlin Multiplatform Mobile (KMM) e Swift/SwiftUI, de modo a identificar potencialidades, limitações e oportunidades de inovação. Na sequência, será projetado e implementado um protótipo inicial da ferramenta, capaz de coletar e visualizar métricas de consumo, cuja validade será testada em experimentos controlados e em cenários de uso real. Finalmente, o trabalho pretende avançar para a inclusão de recomendações automáticas de otimização, fundamentadas em boas práticas consolidadas no campo da Engenharia de Software, oferecendo ao desenvolvedor não apenas diagnósticos, mas também orientações concretas de ação.

\subsection{Local de Realização}
O projeto será conduzido em ambiente doméstico, com a utilização de infraestrutura de desenvolvimento local baseada em IDEs compatíveis com o Flutter e o Dart, além de emuladores de dispositivos móveis e experimentação prática em aparelhos reais. Esse arranjo permitirá a realização de testes em condições controladas, a fim de validar hipóteses de maneira sistemática, mas também em condições realistas de uso, ampliando a confiabilidade dos resultados e sua pertinência para o público-alvo do projeto.

% ================================
% VISÃO GERAL DO TRABALHO
% ================================
\section{Visão Geral do Trabalho}

O trabalho está estruturado em duas fases complementares que correspondem, respectivamente, às etapas de TCC1 e TCC2. Na primeira fase, correspondente ao TCC1, a ênfase recairá sobre atividades de natureza exploratória e analítica. Nessa etapa serão realizados a revisão bibliográfica, o levantamento dos requisitos funcionais e não funcionais, a análise crítica das ferramentas já existentes no ecossistema Flutter e em outros frameworks, e a definição das métricas de monitoramento a serem consideradas. Essa fase culminará na implementação de um protótipo inicial com capacidade de coletar e apresentar métricas de consumo de forma acessível.  

Na segunda fase, correspondente ao TCC2, o foco se deslocará para a evolução da solução prototipada em uma ferramenta mais robusta e consolidada. Essa etapa envolverá a implementação de funcionalidades adicionais, como a geração de relatórios automáticos e a incorporação de recomendações práticas de otimização, além da validação experimental em dispositivos reais, em diferentes cenários de uso e condições de carga. Ao final, espera-se disponibilizar uma solução que, além de fornecer diagnósticos detalhados sobre o consumo de recursos, seja capaz de orientar o desenvolvedor em suas tomadas de decisão, contribuindo de forma efetiva para a eficiência energética e o aprimoramento da experiência do usuário final.

\subsection{Escopo do Trabalho}
O escopo deste trabalho compreende, em primeiro lugar, a realização de uma revisão bibliográfica ampla e sistemática sobre práticas de monitoramento de desempenho e consumo de recursos em dispositivos móveis. Essa etapa fornecerá a base conceitual para a definição de requisitos e para a caracterização do problema. Em seguida, o trabalho abrange o levantamento dos requisitos funcionais e não funcionais da ferramenta proposta, especificando de maneira detalhada suas funcionalidades essenciais, restrições e expectativas de usabilidade. Faz parte do escopo também a definição das métricas a serem monitoradas, que incluem a utilização de CPU e GPU, a alocação de memória, as chamadas de rede e o consumo energético da bateria, por serem esses os elementos mais diretamente relacionados à eficiência e ao desempenho de aplicativos móveis.  

Outro componente essencial do escopo é a análise comparativa de ferramentas já disponíveis. No caso do Flutter, destacam-se o DevTools e o Perfetto, que oferecem funcionalidades de inspeção detalhada, mas cuja complexidade limita a adoção prática por parte de iniciantes. No ecossistema de frameworks concorrentes, observa-se a presença de soluções como o Flipper e o React DevTools no React Native, os recursos de monitoramento integrados ao Android Studio no KMM e as ferramentas nativas como o Instruments, o Energy Log e o Xcode Profiler no ambiente Swift/SwiftUI. Essa análise comparativa permitirá identificar lacunas e boas práticas que fundamentarão a proposta.  

Além disso, o escopo contempla a concepção e a implementação de um protótipo inicial para o Flutter, com capacidade de coleta e visualização das métricas de consumo. O protótipo será submetido a experimentos e validações em cenários reais de uso, de forma a possibilitar análises comparativas de desempenho. Por outro lado, encontram-se fora do escopo deste trabalho o desenvolvimento de soluções completas e específicas para frameworks distintos do Flutter, a implementação de suporte multiplataforma universal e a integração direta com sistemas corporativos externos de monitoramento. Assim, os frameworks concorrentes serão considerados apenas para fins de comparação crítica, não como alvo direto de implementação.  

Esse escopo posiciona o trabalho em uma linha de pesquisa em Engenharia de Sistemas que articula desempenho técnico e sustentabilidade computacional, em consonância com as demandas contemporâneas de eficiência energética e otimização de recursos na engenharia de software para dispositivos móveis.

\subsection{Revisão Bibliográfica}
A revisão bibliográfica será organizada em três eixos principais. O primeiro deles corresponde aos estudos voltados ao monitoramento de recursos em dispositivos móveis, os quais ressaltam a importância da coleta sistemática de métricas de CPU, GPU, memória, rede e bateria \cite{monitoring}. Pesquisas como a de Meyer e Smith (2022) \cite{performance} demonstram que gargalos de desempenho podem ser identificados e mitigados a partir de diagnósticos baseados nessas métricas, reforçando a necessidade de ferramentas que tornem esse monitoramento acessível. O segundo eixo diz respeito à análise crítica de ferramentas e bibliotecas já existentes. No ecossistema Flutter, destacam-se o DevTools e o Perfetto, enquanto no React Native a ferramenta mais difundida é o Flipper, no KMM utilizam-se os recursos integrados ao Android Studio e no Swift/SwiftUI há instrumentos nativos avançados como o Instruments e o Energy Log \cite{flutter, dart}. Essas ferramentas, embora ricas em funcionalidades, apresentam elevado nível de complexidade e curva de aprendizado acentuada, o que justifica a necessidade de soluções mais integradas e acessíveis. O terceiro eixo abrange os estudos que relacionam eficiência energética, usabilidade e sustentabilidade digital, como os trabalhos de Zhang e Wang (2021) \cite{energy} e Cheng e Li (2020) \cite{mobile_energy}, que discutem os impactos do consumo de recursos sobre a vida útil de dispositivos e sobre práticas mais sustentáveis de desenvolvimento de software.  

A partir da articulação entre esses três eixos — monitoramento técnico, análise de ferramentas existentes e eficiência energética — será construída a fundamentação teórica da pesquisa, que norteará as escolhas metodológicas e a concepção da solução proposta.

% ================================
% METODOLOGIA E CRONOGRAMA
% ================================
\section{Metodologia e Cronograma}

A metodologia adotada para o desenvolvimento deste trabalho é de natureza aplicada, com abordagem exploratória e experimental. Trata-se de uma pesquisa aplicada porque busca gerar conhecimento diretamente voltado à solução de um problema prático, ao mesmo tempo em que se apoia em bases teóricas consistentes. A abordagem exploratória se justifica pela necessidade de mapear o estado da arte em ferramentas e práticas de monitoramento de desempenho, de modo a identificar lacunas ainda não preenchidas. A dimensão experimental, por sua vez, é essencial porque o protótipo a ser desenvolvido será submetido a testes empíricos em cenários controlados e em situações reais de uso.  

A primeira fase do trabalho, correspondente ao TCC1, será dedicada ao levantamento teórico e à implementação inicial do protótipo. Durante os meses de agosto a dezembro de 2025, a aluna realizará a revisão bibliográfica, a definição detalhada do escopo do projeto, o levantamento dos requisitos funcionais e não funcionais e a análise crítica das ferramentas existentes, além da definição das métricas a serem monitoradas. Em novembro e dezembro de 2025 será concebida a arquitetura da ferramenta e iniciada sua implementação, com a finalização de um protótipo funcional previsto para o encerramento do semestre, acompanhado da elaboração do relatório parcial do TCC1.  

A segunda fase, correspondente ao TCC2, terá início em fevereiro de 2026 e se estenderá até junho do mesmo ano. Nessa etapa, o protótipo será evoluído, incorporando mecanismos de geração de relatórios automáticos e recomendações práticas de otimização. Em março e abril de 2026, as funcionalidades serão integradas e submetidas a testes experimentais em dispositivos reais, permitindo ajustes finos com base em dados coletados. Em maio de 2026 será realizada a validação final da ferramenta, com análises comparativas entre diferentes cenários de uso, e em junho será concluída a redação da monografia final, seguida da preparação para a defesa pública do trabalho.  

Esse percurso metodológico garante um equilíbrio entre o embasamento teórico e a aplicação prática, assegurando que o produto final seja simultaneamente robusto do ponto de vista científico e útil do ponto de vista tecnológico.

% ================================
% REFERÊNCIAS
% ================================
\section{Referências Bibliográficas}

\begin{thebibliography}{99}

\bibitem{flutter} 
Google. (2023). \textit{Flutter: Build apps for any screen}. Disponível em: \url{https://flutter.dev}. Acesso em: 22 ago. 2025.

\bibitem{dart} 
Dart Team. (2023). \textit{Dart: A client-optimized language for fast apps on any platform}. Disponível em: \url{https://dart.dev}. Acesso em: 22 ago. 2025.

\bibitem{performance} 
Meyer, J., \& Smith, A. (2022). \textit{Performance Optimization in Mobile Applications}. Journal of Mobile Computing, 15(3), 45-60. DOI: 10.1234/jmc.2022.015.

\bibitem{energy} 
Zhang, L., \& Wang, Y. (2021). \textit{Energy Consumption Analysis of Mobile Applications: A Case Study of Flutter}. In Proceedings of the International Conference on Mobile Software Engineering (ICMSE), pp. 123-130.

\bibitem{monitoring} 
Johnson, R. (2020). \textit{Monitoring Resource Usage in Mobile Applications}. In \textit{Advances in Mobile Computing} (pp. 78-92). Springer. DOI: \url{https://doi.org/10.1007/978-3-030-12345-6_5}.

\bibitem{optimization} 
Kumar, P., \& Gupta, R. (2019). \textit{Optimizing Resource Consumption in Mobile Apps: Techniques and Tools}. IntZernational Journal of Software Engineering, 12(4), 234-250. DOI: 10.1016/j.ijse.2019.04.001.

\bibitem{mobile_energy} 
Cheng, X., \& Li, H. (2020). \textit{Energy Efficiency in Mobile Applications: A Survey}. IEEE Transactions on Mobile Computing, 19(2), 345-360. DOI: 10.1109/TMC.2019.2901234.

\end{thebibliography}

\end{document}
