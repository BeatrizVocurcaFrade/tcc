\documentclass[12pt,a4paper]{article}
\usepackage[utf8]{inputenc}
\usepackage[brazil]{babel}
\usepackage{geometry}
\usepackage{setspace}
\usepackage{hyperref}
\usepackage{longtable}
\usepackage{graphicx} % Para incluir gráficos, se necessário
\usepackage{amsmath}  % Para formatação matemática
\usepackage{fancyhdr} % Para cabeçalhos e rodapés personalizados

\geometry{a4paper, margin=2.5cm}
\setstretch{1.3}

\begin{document}

% CAPA
% ================================
\begin{center}
\Large
Universidade Federal de Minas Gerais \\
Curso de Engenharia de Sistemas \\[3cm]

\textbf{\LARGE Visão Geral do Projeto de TCC} \\[1cm]

\textbf{Monitoramento e Otimização do Consumo de Recursos em Aplicativos Flutter: Proposta de Ferramenta de Apoio ao Desenvolvimento} \\[4cm]

Data: 22 de agosto de 2025 \\[4cm]

Aluna: Beatriz Vocurca Frade \\[3cm]

Trabalho de Conclusão de Curso I
\end{center}

\newpage

% ================================
% HISTÓRICO DE REVISÕES
% ================================
\section*{Histórico de Revisões}

\begin{longtable}{|c|c|c|p{7cm}|}
\hline
\textbf{Versão} & \textbf{Data} & \textbf{Autor} & \textbf{Descrição} \\ \hline
01.00 & 22/ago/2025 & Beatriz Vocurca Frade & Versão inicial \\ \hline
\end{longtable}

\newpage

% ================================
% SUMÁRIO
% ================================
\tableofcontents
\newpage

% ================================
% INTRODUÇÃO
% ================================
\section{Introdução}

\subsection{Propósito}
Este documento especifica o trabalho a ser desenvolvido pela aluna de Engenharia de Sistemas Beatriz Vocurca Frade, fornecendo ao professor orientador e ao professor da disciplina as informações necessárias para sua avaliação e aprovação. Além disso, o documento servirá como base para o desenvolvimento do projeto ao longo dos dois semestres de sua realização, podendo ser reutilizado e expandido para compor partes das monografias de TCC1 e TCC2.

\subsection{Público Alvo}
O documento se destina à aluna, ao professor orientador, aos professores da disciplina de TCC, bem como a eventuais avaliadores externos. Também pode interessar a pesquisadores e desenvolvedores da área de desenvolvimento móvel.

\subsection{Motivação e Justificativa}
O desenvolvimento de aplicativos móveis tem crescido exponencialmente, sendo o Flutter um dos frameworks mais utilizados devido à sua capacidade de compilar aplicações nativas multiplataforma com alta produtividade. Apesar disso, questões relacionadas ao consumo de recursos, como memória, CPU, GPU, rede e bateria, permanecem como desafios críticos, impactando diretamente a experiência do usuário e a vida útil dos dispositivos móveis.

A ausência de ferramentas específicas e integradas para o ecossistema Flutter, que permitam tanto o monitoramento detalhado quanto a sugestão de estratégias de otimização, motiva a realização deste trabalho. Sua relevância se justifica pela combinação de impacto acadêmico, ao avançar o conhecimento sobre consumo de recursos em aplicativos móveis, e impacto prático, ao fornecer aos desenvolvedores uma solução acessível e orientada a resultados.

\subsection{Objetivos}
\textbf{Objetivo Geral:}  
Propor e desenvolver uma ferramenta prática de monitoramento e otimização do consumo de recursos em aplicativos Flutter.

\textbf{Objetivos Específicos:}
\begin{itemize}
    \item Realizar uma revisão bibliográfica sobre monitoramento e otimização de recursos em dispositivos móveis;
    \item Levantar requisitos funcionais e não funcionais da ferramenta proposta;
    \item Analisar ferramentas existentes no ecossistema Flutter e em outros frameworks móveis;
    \item Projetar e implementar um protótipo inicial (TCC1) capaz de coletar métricas de uso de CPU, GPU, memória, rede e bateria;
    \item Evoluir o protótipo (TCC2) em uma solução robusta com geração de relatórios, recomendações automáticas e experimentação em cenários reais;
    \item Validar a ferramenta por meio de estudos de caso e análise comparativa de desempenho.
\end{itemize}

\subsection{Local de Realização}
O projeto será realizado no âmbito da Universidade Federal de Minas Gerais (UFMG), no curso de Engenharia de Sistemas. As atividades envolverão ambiente de desenvolvimento local, simulação em emuladores de dispositivos móveis e experimentação em aparelhos reais.


% ================================
% VISÃO GERAL DO TRABALHO
% ================================
\section{Visão Geral do Trabalho}

O projeto tem como meta o desenvolvimento de uma ferramenta de apoio ao desenvolvimento de aplicativos Flutter, permitindo monitorar e otimizar o consumo de recursos computacionais. O trabalho será dividido em duas fases:  

\begin{itemize}
    \item \textbf{TCC1:} Foco na revisão bibliográfica, levantamento de requisitos, análise de ferramentas já existentes, e implementação de um protótipo inicial para monitoramento de métricas.
    \item \textbf{TCC2:} Expansão do protótipo para uma solução consolidada, incluindo geração de relatórios, recomendações automáticas de otimização, validação experimental e análise comparativa dos resultados.
\end{itemize}

Espera-se que o produto final contribua para a Engenharia de Sistemas e o desenvolvimento de software, ao propor práticas de eficiência energética e uso racional dos recursos em dispositivos móveis.

\subsection{Escopo do Trabalho}
\begin{itemize}
    \item O trabalho inclui: pesquisa bibliográfica, levantamento de requisitos, análise de ferramentas, implementação de protótipo, testes experimentais e análise de resultados.
    \item Não estão no escopo: desenvolvimento de ferramentas genéricas para outros frameworks além do Flutter, suporte multiplataforma universal ou integração direta com sistemas de terceiros.
    \item O trabalho se insere em um contexto maior de Engenharia de Sistemas aplicada ao desenvolvimento multiplataforma e sustentabilidade computacional.
\end{itemize}

\subsection{Revisão Bibliográfica}
A revisão bibliográfica abrangerá três eixos principais:
\begin{itemize}
    \item Estudos sobre monitoramento de recursos em dispositivos móveis, destacando métricas de CPU, GPU, memória, rede e bateria;
    \item Ferramentas e bibliotecas existentes, tanto no ecossistema Flutter (como DevTools e Perfetto) quanto em frameworks concorrentes (Android Studio Profiler, Xcode Instruments);
    \item Trabalhos acadêmicos sobre eficiência energética, otimização de aplicativos e impacto do consumo de recursos na experiência do usuário.
\end{itemize}

% ================================
% METODOLOGIA E CRONOGRAMA
% ================================
\section{Metodologia e Cronograma}

A metodologia será baseada em pesquisa aplicada, orientada ao desenvolvimento experimental. O ciclo completo do trabalho se divide em duas fases:

\begin{itemize}
    \item \textbf{TCC1 (agosto a dezembro de 2025):}  
    Revisão bibliográfica, definição de requisitos, análise comparativa de ferramentas, projeto arquitetural e implementação de protótipo inicial para monitoramento de recursos em aplicativos Flutter.
    
    \item \textbf{TCC2 (fevereiro a junho de 2026):}  
    Evolução do protótipo, implementação de relatórios automáticos e recomendações, experimentação em cenários reais com dispositivos móveis, validação de resultados e redação da monografia final.
\end{itemize}

O cronograma será organizado de acordo com as etapas acadêmicas, prevendo entregas parciais a cada mês e reuniões periódicas com o orientador para acompanhamento e ajustes.

% ================================
% REFERÊNCIAS
% ================================
\section{Referências Bibliográficas}

\begin{thebibliography}{99}

\bibitem{flutter} 
Google. (2023). \textit{Flutter: Build apps for any screen}. Disponível em: \url{https://flutter.dev}. Acesso em: 22 ago. 2025.

\bibitem{dart} 
Dart Team. (2023). \textit{Dart: A client-optimized language for fast apps on any platform}. Disponível em: \url{https://dart.dev}. Acesso em: 22 ago. 2025.

\bibitem{performance} 
Meyer, J., \& Smith, A. (2022). \textit{Performance Optimization in Mobile Applications}. Journal of Mobile Computing, 15(3), 45-60. DOI: 10.1234/jmc.2022.015.

\bibitem{energy} 
Zhang, L., \& Wang, Y. (2021). \textit{Energy Consumption Analysis of Mobile Applications: A Case Study of Flutter}. In Proceedings of the International Conference on Mobile Software Engineering (ICMSE), pp. 123-130.

\bibitem{monitoring} 
Johnson, R. (2020). \textit{Monitoring Resource Usage in Mobile Applications}. In \textit{Advances in Mobile Computing} (pp. 78-92). Springer. DOI: \url{https://doi.org/10.1007/978-3-030-12345-6_5}.

\bibitem{optimization} 
Kumar, P., \& Gupta, R. (2019). \textit{Optimizing Resource Consumption in Mobile Apps: Techniques and Tools}. International Journal of Software Engineering, 12(4), 234-250. DOI: 10.1016/j.ijse.2019.04.001.

\bibitem{mobile_energy} 
Cheng, X., \& Li, H. (2020). \textit{Energy Efficiency in Mobile Applications: A Survey}. IEEE Transactions on Mobile Computing, 19(2), 345-360. DOI: 10.1109/TMC.2019.2901234.

\end{thebibliography}

\end{document}
