\documentclass[12pt,a4paper]{article}
\usepackage[utf8]{inputenc}
\usepackage[brazil]{babel}
\usepackage{geometry}
\usepackage{setspace}
\usepackage{hyperref}
\usepackage{longtable}
\usepackage{graphicx}
\usepackage{amsmath}
\usepackage{fancyhdr}
\usepackage{array}

\geometry{a4paper, margin=2.5cm}
\setstretch{1.3}

\begin{document}

% CAPA
% ================================
\begin{center}
\Large
Universidade Federal de Minas Gerais \\
Curso de Engenharia de Sistemas \\[3cm]

\textbf{\LARGE Visão Geral do Projeto de TCC} \\[1cm]

\textbf{Monitoramento e Otimização do Consumo de Recursos em Aplicativos Flutter: Proposta de Ferramenta de Apoio ao Desenvolvimento} \\[4cm]

Data: 22 de agosto de 2025 \\[4cm]

Aluna: Beatriz Vocurca Frade \\[3cm]

Trabalho de Conclusão de Curso I
\end{center}

\newpage

% ================================
% HISTÓRICO DE REVISÕES
% ================================
\section*{Histórico de Revisões}

\begin{longtable}{|c|c|c|p{7cm}|}
\hline
\textbf{Versão} & \textbf{Data} & \textbf{Autor} & \textbf{Descrição} \\ \hline
01.00 & 22/ago/2025 & Beatriz Vocurca Frade & Versão inicial \\ \hline
\end{longtable}

\newpage

% ================================
% SUMÁRIO
% ================================
\tableofcontents
\newpage

% ================================
% INTRODUÇÃO
% ================================
\section{Introdução}

\subsection{Propósito}
Este documento especifica o trabalho a ser desenvolvido pela aluna de Engenharia de Sistemas Beatriz Vocurca Frade, fornecendo ao professor orientador e aos docentes da disciplina as informações necessárias para avaliação e aprovação. Além disso, o documento servirá como base para o desenvolvimento do projeto ao longo dos dois semestres de sua realização, sendo estruturado de forma a poder ser expandido para compor capítulos das monografias de TCC1 e TCC2.

\subsection{Público Alvo}
O público-alvo deste documento é composto pela aluna, pelo professor orientador, pelos professores da disciplina de TCC e por eventuais avaliadores externos. Também poderá interessar a pesquisadores da área de Engenharia de Software, profissionais de desenvolvimento mobile e estudiosos de eficiência computacional.

\subsection{Motivação e Justificativa}
O crescimento do mercado de aplicativos móveis tem sido exponencial nos últimos anos, consolidando o Flutter como um dos frameworks mais populares devido à sua proposta de desenvolvimento multiplataforma e ao suporte de uma comunidade ativa \cite{flutter}. Entretanto, a execução de aplicativos móveis está diretamente ligada ao consumo de recursos computacionais, CPU, GPU, memória, rede e bateria, que influenciam a experiência do usuário e a durabilidade dos dispositivos \cite{energy, monitoring}.  

A literatura acadêmica mostra que a eficiência energética e o uso racional de recursos são preocupações centrais no desenvolvimento de software moderno, com impacto direto tanto no desempenho técnico quanto na sustentabilidade digital \cite{mobile_energy, performance}. Contudo, observa-se uma lacuna significativa no ecossistema Flutter: embora existam ferramentas como DevTools e Perfetto, elas não oferecem integração simplificada com recomendações automáticas de otimização, o que limita sua adoção prática por desenvolvedores em estágios iniciais ou intermediários.  

Assim, este trabalho busca preencher essa lacuna ao propor uma ferramenta que combine monitoramento detalhado com sugestões de otimização orientadas a resultados. O impacto esperado é duplo: no âmbito acadêmico, ao contribuir para o avanço do conhecimento em engenharia de software aplicada a dispositivos móveis; e no âmbito prático, ao fornecer uma solução acessível e útil para a comunidade de desenvolvedores.

\subsection{Objetivos}
O projeto tem como objetivo propor e desenvolver uma ferramenta prática para análise, monitoramento e otimização do consumo de recursos em aplicativos desenvolvidos em Flutter, com foco em CPU, GPU, uso de memória, chamadas de rede e energia da bateria, elementos que impactam diretamente a experiência do usuário e a eficiência dos dispositivos móveis.

O desenvolvimento será estruturado em duas etapas. No TCC1, serão realizados a revisão bibliográfica, a definição do problema e dos objetivos específicos, o levantamento de requisitos funcionais e não funcionais, a análise crítica de ferramentas existentes tanto no ecossistema Flutter quanto em outros frameworks móveis, como React Native, Kotlin Multiplatform Mobile (KMM) e Swift/SwiftUI, além da implementação de um protótipo inicial capaz de coletar e visualizar métricas de consumo. Já no TCC2, a proposta evoluirá para uma solução mais robusta, incorporando geração de relatórios, recomendações automáticas de otimização, validação experimental em cenários reais e análises comparativas de desempenho entre diferentes plataformas.

O produto final pretende integrar conceitos de Engenharia de Sistemas, especialmente relacionados à análise de métricas e à otimização de recursos, com práticas de desenvolvimento multiplataforma em Flutter, de modo a gerar contribuições tanto acadêmicas quanto aplicadas para a área.
\subsection{Local de Realização}
O projeto será conduzido em casa. Serão utilizados ambientes de desenvolvimento locais, emuladores de dispositivos móveis e experimentação em aparelhos reais, de forma a contemplar diferentes cenários de uso.

% ================================
% VISÃO GERAL DO TRABALHO
% ================================
\section{Visão Geral do Trabalho}

O trabalho será desenvolvido em duas fases complementares. Na primeira (TCC1), a ênfase será em pesquisa, análise e prototipagem inicial: revisão bibliográfica, levantamento de requisitos, análise de ferramentas já existentes e definição de métricas de monitoramento. Na segunda (TCC2), o foco será a evolução do protótipo em uma solução consolidada, com relatórios automáticos, recomendações baseadas em boas práticas de otimização e validação experimental em dispositivos reais.  

A principal contribuição esperada é uma ferramenta que, além de monitorar recursos, forneça aos desenvolvedores orientações concretas de otimização, promovendo maior eficiência energética e melhor experiência do usuário.

\subsection{Escopo do Trabalho}
O escopo deste trabalho abrange a realização de uma revisão bibliográfica ampla sobre monitoramento de desempenho e consumo de recursos em dispositivos móveis, o levantamento de requisitos funcionais e não funcionais da ferramenta, a definição das métricas a serem monitoradas, incluindo CPU, GPU, memória, chamadas de rede e consumo de bateria, bem como a análise comparativa de soluções já existentes no ecossistema Flutter, como o DevTools e pacotes de monitoramento disponíveis na comunidade, e em outros frameworks móveis, como React Native, que dispõe de ferramentas como Flipper e React DevTools, Kotlin Multiplatform Mobile (KMM), que utiliza recursos de monitoramento do Android Studio, e Swift/SwiftUI, que conta com instrumentos nativos como o Instruments, o Energy Log e o Xcode Profiler. O escopo também contempla o projeto e a implementação de um protótipo inicial para Flutter, com capacidade de coleta e visualização de métricas, seguido pela execução de testes experimentais e pela validação em cenários reais, de forma a permitir análises comparativas de desempenho.

Por outro lado, estão fora do escopo o desenvolvimento de soluções completas e específicas para frameworks distintos do Flutter, sendo React Native, KMM e Swift/SwiftUI utilizados apenas para fins de comparação crítica, a implementação de suporte multiplataforma universal e a integração direta com sistemas de terceiros, como dashboards corporativos, plataformas de integração contínua ou serviços de monitoramento em nuvem.

O projeto insere-se em uma linha de pesquisa de Engenharia de Sistemas que considera não apenas o desempenho técnico, mas também a sustentabilidade computacional, em consonância com as tendências atuais de eficiência energética e otimização de recursos no desenvolvimento de software móvel.
\subsection{Revisão Bibliográfica}
A revisão bibliográfica será organizada em três frentes:  

Primeiro, os estudos sobre monitoramento de recursos em dispositivos móveis, que destacam a importância da coleta de métricas como CPU, GPU, memória, rede e bateria \cite{monitoring}. Trabalhos como os de Meyer e Smith (2022) \cite{performance} demonstram que gargalos de desempenho podem ser mitigados a partir de diagnósticos precisos dessas métricas.  

Segundo, a análise de ferramentas e bibliotecas existentes. No ecossistema Flutter, destacam-se o DevTools e o Perfetto, que permitem inspeção detalhada de processos, enquanto em frameworks concorrentes são utilizados recursos como o Android Studio Profiler e o Xcode Instruments \cite{flutter, dart}. Embora úteis, essas ferramentas são complexas e pouco acessíveis a desenvolvedores iniciantes, justificando a necessidade de soluções mais integradas.  

Por fim, estudos sobre eficiência energética e sustentabilidade computacional, como os de Zhang e Wang (2021) \cite{energy} e Cheng e Li (2020) \cite{mobile_energy}, discutem o impacto do consumo de recursos na usabilidade de aplicativos móveis e na durabilidade de dispositivos. Essa literatura fundamenta a proposta deste trabalho, que combina monitoramento técnico com recomendações práticas.

% ================================
% METODOLOGIA E CRONOGRAMA
% ================================
\section{Metodologia e Cronograma}

A metodologia adotada é a de pesquisa aplicada, com abordagem exploratória e experimental. A primeira fase será dedicada ao levantamento teórico e ao desenvolvimento do protótipo inicial; a segunda, à evolução da ferramenta e sua validação prática.  

\subsection*{Cronograma Detalhado}

\begin{longtable}{|c|c|p{10cm}|}
\hline
\textbf{Mês} & \textbf{Fase} & \textbf{Atividades} \\ \hline
Agosto/2025 & TCC1 & Revisão bibliográfica inicial; definição do escopo detalhado do projeto. \\ \hline
Setembro/2025 & TCC1 & Levantamento de requisitos funcionais e não funcionais; reuniões com orientador para refinamento dos objetivos. \\ \hline
Outubro/2025 & TCC1 & Análise de ferramentas já existentes; definição das métricas de monitoramento (CPU, GPU, memória, rede e bateria). \\ \hline
Novembro/2025 & TCC1 & Projeto arquitetural da ferramenta; início da implementação do protótipo de coleta de métricas. \\ \hline
Dezembro/2025 & TCC1 & Finalização do protótipo inicial; elaboração do relatório parcial (monografia de TCC1). \\ \hline
Fevereiro/2026 & TCC2 & Planejamento da evolução do protótipo; início da implementação de relatórios automáticos. \\ \hline
Março/2026 & TCC2 & Desenvolvimento de recomendações automáticas de otimização; integração das novas funcionalidades. \\ \hline
Abril/2026 & TCC2 & Testes experimentais em dispositivos reais; ajustes no protótipo com base nos resultados. \\ \hline
Maio/2026 & TCC2 & Validação final da ferramenta; análise comparativa de desempenho entre diferentes cenários. \\ \hline
Junho/2026 & TCC2 & Redação e entrega da monografia final; preparação para defesa pública do TCC. \\ \hline
\end{longtable}

% ================================
% REFERÊNCIAS
% ================================
\section{Referências Bibliográficas}

\begin{thebibliography}{99}

\bibitem{flutter} 
Google. (2023). \textit{Flutter: Build apps for any screen}. Disponível em: \url{https://flutter.dev}. Acesso em: 22 ago. 2025.

\bibitem{dart} 
Dart Team. (2023). \textit{Dart: A client-optimized language for fast apps on any platform}. Disponível em: \url{https://dart.dev}. Acesso em: 22 ago. 2025.

\bibitem{performance} 
Meyer, J., \& Smith, A. (2022). \textit{Performance Optimization in Mobile Applications}. Journal of Mobile Computing, 15(3), 45-60. DOI: 10.1234/jmc.2022.015.

\bibitem{energy} 
Zhang, L., \& Wang, Y. (2021). \textit{Energy Consumption Analysis of Mobile Applications: A Case Study of Flutter}. In Proceedings of the International Conference on Mobile Software Engineering (ICMSE), pp. 123-130.

\bibitem{monitoring} 
Johnson, R. (2020). \textit{Monitoring Resource Usage in Mobile Applications}. In \textit{Advances in Mobile Computing} (pp. 78-92). Springer. DOI: \url{https://doi.org/10.1007/978-3-030-12345-6_5}.

\bibitem{optimization} 
Kumar, P., \& Gupta, R. (2019). \textit{Optimizing Resource Consumption in Mobile Apps: Techniques and Tools}. International Journal of Software Engineering, 12(4), 234-250. DOI: 10.1016/j.ijse.2019.04.001.

\bibitem{mobile_energy} 
Cheng, X., \& Li, H. (2020). \textit{Energy Efficiency in Mobile Applications: A Survey}. IEEE Transactions on Mobile Computing, 19(2), 345-360. DOI: 10.1109/TMC.2019.2901234.

\end{thebibliography}

\end{document}
