\documentclass[12pt,a4paper]{article}
\usepackage[utf8]{inputenc}
\usepackage[brazil]{babel}
\usepackage{geometry}
\usepackage{setspace}
\usepackage{hyperref}
\usepackage{longtable}
\usepackage{graphicx}
\usepackage{amsmath}
\usepackage{fancyhdr}
\usepackage{array}

\geometry{a4paper, margin=2.5cm}
\setstretch{1.3}

\begin{document}

% CAPA
% ================================
\begin{center}
\Large
Universidade Federal de Minas Gerais \\
Curso de Engenharia de Sistemas \\[3cm]

\textbf{\LARGE Visão Geral do Projeto de TCC} \\[1cm]

\textbf{Monitoramento e Otimização do Consumo de Recursos em Aplicativos Flutter: Proposta de Ferramenta de Apoio ao Desenvolvimento} \\[4cm]

Data: 22 de agosto de 2025 \\[4cm]

Aluna: Beatriz Vocurca Frade \\[3cm]

Trabalho de Conclusão de Curso I
\end{center}

\newpage

% ================================
% HISTÓRICO DE REVISÕES
% ================================
\section*{Histórico de Revisões}

\begin{longtable}{|c|c|c|p{7cm}|}
\hline
\textbf{Versão} & \textbf{Data} & \textbf{Autor} & \textbf{Descrição} \\ \hline
01.00 & 22/ago/2025 & Beatriz Vocurca Frade & Versão inicial \\ \hline
\end{longtable}

\newpage

% ================================
% SUMÁRIO
% ================================
\tableofcontents
\newpage

% ================================
% INTRODUÇÃO
% ================================
\section{Introdução}

\subsection{Propósito}
Este documento apresenta e especifica o trabalho de conclusão de curso a ser desenvolvido pela aluna Beatriz Vocurca Frade no âmbito do curso de Engenharia de Sistemas da Universidade Federal de Minas Gerais. Tem como finalidade estabelecer uma base formal para o desenvolvimento da pesquisa, servindo como instrumento de avaliação inicial por parte do professor orientador e dos docentes da disciplina de TCC, além de constituir um guia estruturado que permitirá a expansão dos conteúdos aqui delineados em capítulos futuros das monografias correspondentes ao TCC1 e ao TCC2. O texto busca explicitar objetivos, delimitações e a relevância acadêmica e prática do trabalho, deixando clara a contribuição científica e tecnológica da proposta.

\subsection{Público-Alvo}
O público-alvo imediato inclui a própria aluna responsável pelo desenvolvimento, o professor orientador e os docentes vinculados à disciplina de TCC, bem como os avaliadores internos e externos responsáveis pela análise do projeto. Em sentido ampliado, a pesquisa interessa a pesquisadores de Engenharia de Software, profissionais do desenvolvimento mobile, engenheiros de sistemas envolvidos em aplicações críticas de consumo energético e estudiosos de eficiência computacional e sustentabilidade digital. Assim, embora seja orientado a um contexto acadêmico, o conteúdo possui relevância prática e dialoga com a comunidade técnica de desenvolvedores de software.

\subsection{Motivação e Justificativa}
O crescimento exponencial de aplicativos móveis ao longo da última década representa uma das principais vertentes da inovação tecnológica contemporânea. Nesse contexto, o Flutter consolidou-se como um framework amplamente utilizado, permitindo o desenvolvimento multiplataforma com desempenho elevado e contando com suporte de uma comunidade ativa \cite{flutter}. Entretanto, a execução de aplicativos móveis envolve consumo de recursos computacionais fundamentais, como CPU, GPU, memória, rede e bateria, que impactam diretamente a experiência do usuário e o desempenho dos dispositivos \cite{energy, monitoring}.  

A literatura acadêmica discute amplamente a necessidade de eficiência energética e uso racional de recursos como parte integrante do desenvolvimento de software moderno \cite{mobile_energy, performance}. Essa preocupação conecta-se a questões de sustentabilidade digital, uma vez que a redução do consumo energético em escala global contribui para a diminuição da pegada de carbono gerada por sistemas computacionais. Apesar da relevância do tema, o ecossistema Flutter carece de soluções integradas que combinem monitoramento detalhado do uso de recursos com recomendações práticas de otimização. Ferramentas existentes, como DevTools e Perfetto, embora poderosas, apresentam barreiras de adoção por serem complexas, limitando sua aplicação em projetos cotidianos.  

O presente trabalho busca preencher essa lacuna propondo uma ferramenta que una monitoramento de métricas de consumo de recursos a sugestões automáticas de otimização. A relevância da proposta se manifesta em duas dimensões: acadêmica, ao ampliar o conhecimento em Engenharia de Software aplicada a dispositivos móveis, e prática, ao disponibilizar uma solução acessível à comunidade de desenvolvedores, democratizando práticas avançadas de análise de eficiência computacional.

\subsection{Objetivos}
O objetivo geral deste trabalho é desenvolver uma ferramenta capaz de analisar, monitorar e otimizar o consumo de recursos em aplicativos Flutter, considerando CPU, GPU, memória, rede e bateria. A ferramenta deverá fornecer informações detalhadas sobre o desempenho do aplicativo e propor recomendações práticas para aumentar a eficiência dos dispositivos e melhorar a experiência do usuário.

Para atingir esse objetivo, inicialmente será realizada uma revisão bibliográfica sistemática sobre monitoramento de desempenho, consumo de recursos e boas práticas de otimização em aplicativos móveis, estabelecendo a base teórica e situando o estado da arte. Em seguida, será realizado o levantamento detalhado dos requisitos da ferramenta, incluindo coleta e visualização de métricas, geração de relatórios estruturados e fornecimento de recomendações automáticas de otimização. Também serão definidos critérios não funcionais, como eficiência da ferramenta para não interferir no desempenho do aplicativo, compatibilidade com múltiplos dispositivos, interface intuitiva e confiabilidade das métricas.

O trabalho inclui ainda a análise crítica de ferramentas existentes, como Flutter DevTools, Perfetto, Android Studio Profiler e Xcode Instruments, bem como de recursos de frameworks concorrentes, avaliando funcionalidades, limitações e possibilidades de integração. Com base nessa análise, será projetado e implementado um protótipo capaz de coletar e visualizar métricas em tempo real e gerar relatórios, testado em cenários controlados e em dispositivos reais para validar precisão e aplicabilidade.

Por fim, o desenvolvimento avançará para inclusão de recomendações automáticas de otimização fundamentadas em boas práticas consolidadas de Engenharia de Software, de modo que o desenvolvedor receba diagnósticos e orientações concretas para reduzir consumo de recursos, melhorar performance e aumentar eficiência energética.

\subsection{Local de Realização}
O projeto será conduzido em ambiente doméstico, utilizando infraestrutura de desenvolvimento local baseada em IDEs compatíveis com Flutter e Dart, emuladores de dispositivos móveis e experimentação prática em aparelhos reais. Esse arranjo permitirá testes em condições controladas e realistas, ampliando a confiabilidade dos resultados e sua pertinência para o público-alvo.

% ================================
% VISÃO GERAL DO TRABALHO
% ================================
\section{Visão Geral do Trabalho}
O trabalho está estruturado em duas fases correspondentes às etapas de TCC1 e TCC2. Na primeira fase, TCC1, a ênfase será exploratória e analítica, incluindo revisão bibliográfica, levantamento de requisitos, análise crítica de ferramentas existentes no ecossistema Flutter e em outros frameworks, e definição de métricas de monitoramento. Essa fase culminará na implementação de um protótipo inicial com capacidade de coletar e apresentar métricas de consumo de forma acessível.  

Na segunda fase, TCC2, o foco será a evolução do protótipo para uma ferramenta mais robusta, com funcionalidades adicionais como geração de relatórios automáticos e incorporação de recomendações práticas de otimização, além de validação experimental em dispositivos reais sob diferentes cenários de uso. Ao final, espera-se disponibilizar uma solução capaz de fornecer diagnósticos detalhados e orientar decisões de desenvolvedores, promovendo eficiência energética e aprimoramento da experiência do usuário.

\subsection{Escopo do Trabalho}
O escopo deste trabalho abrange inicialmente a realização de uma revisão bibliográfica extensa sobre práticas de monitoramento de desempenho e consumo de recursos em dispositivos móveis, com o objetivo de fornecer uma base conceitual sólida para a definição dos requisitos da ferramenta e a caracterização detalhada do problema. Nesse contexto, os requisitos da ferramenta incluem tanto aspectos funcionais quanto não funcionais, garantindo que o protótipo atenda às necessidades do desenvolvedor e seja confiável, eficiente e fácil de utilizar. Entre os requisitos funcionais, destacam-se a capacidade de coletar métricas detalhadas de CPU, GPU, memória, chamadas de rede e consumo de bateria, elementos essenciais para avaliar a performance de aplicativos móveis; a apresentação dessas métricas de forma clara e intuitiva por meio de interfaces visuais interativas; a geração de relatórios estruturados que possibilitem análise comparativa de desempenho; e a oferta de recomendações automáticas de otimização, fundamentadas em boas práticas consolidadas de Engenharia de Software.  

Os requisitos não funcionais, por sua vez, incluem a eficiência da ferramenta para que seu funcionamento não interfira significativamente no desempenho do aplicativo monitorado, a compatibilidade com diferentes dispositivos e versões de sistemas operacionais, a confiabilidade e precisão das métricas coletadas, bem como a facilidade de uso e a acessibilidade da interface, de modo a permitir que desenvolvedores iniciantes e intermediários possam extrair valor da ferramenta sem necessidade de conhecimento avançado em monitoramento de performance.  

Além da definição de requisitos, o escopo envolve a análise comparativa de soluções já existentes, tanto no ecossistema Flutter, como o DevTools e o Perfetto, quanto em frameworks concorrentes, como Flipper e React DevTools no React Native, os recursos integrados ao Android Studio no KMM e os instrumentos nativos do Swift/SwiftUI, incluindo Instruments e Energy Log. Essa análise crítica permitirá identificar lacunas, limitações e boas práticas que fundamentarão o projeto da ferramenta, orientando decisões de design, funcionalidades e priorização de métricas.  

Finalmente, o escopo contempla a concepção e implementação de um protótipo inicial voltado para o Flutter, capaz de coletar, processar e apresentar as métricas de consumo de forma visual e interativa, sendo testado em cenários reais de uso para validar sua precisão, eficiência e aplicabilidade. Permanecem fora do escopo deste trabalho o desenvolvimento de soluções completas para frameworks diferentes do Flutter, a implementação de suporte multiplataforma universal e a integração direta com sistemas corporativos externos, sendo estas considerações utilizadas apenas como referência para análise comparativa.


\subsection{Revisão Bibliográfica}
A revisão será organizada em três eixos principais: monitoramento de recursos em dispositivos móveis, análise crítica de ferramentas existentes e estudos sobre eficiência energética, usabilidade e sustentabilidade digital. O primeiro eixo enfatiza coleta sistemática de métricas de CPU, GPU, memória, rede e bateria \cite{monitoring}, destacando a relevância para identificação e mitigação de gargalos de desempenho \cite{performance}. O segundo eixo analisa ferramentas existentes, como Flutter DevTools, Perfetto, Flipper, Android Studio Profiler e instrumentos nativos do Swift/SwiftUI \cite{flutter, dart}, considerando complexidade e acessibilidade. O terceiro eixo aborda impactos do consumo de recursos sobre vida útil de dispositivos e práticas de desenvolvimento sustentáveis \cite{energy, mobile_energy}.  

Essa articulação entre eixos orientará a fundamentação teórica, sustentando escolhas metodológicas e a concepção da solução proposta.

% ================================
% METODOLOGIA E CRONOGRAMA
% ================================
\section{Metodologia e Cronograma}
A metodologia adotada é aplicada, com abordagem exploratória e experimental. Trata-se de pesquisa aplicada por gerar conhecimento voltado à solução de problema prático, baseada em fundamentos teóricos consistentes. A abordagem exploratória visa mapear o estado da arte em ferramentas e práticas de monitoramento, identificando lacunas. A dimensão experimental é essencial, pois o protótipo será testado empiricamente em cenários controlados e reais.

O cronograma detalha atividades semanais do TCC1. Semanas 1 a 3 (04 a 22 de agosto de 2025): revisão bibliográfica aprofundada sobre monitoramento, eficiência energética e otimização, definição detalhada do problema e entrega da Proposta Inicial em 25 de agosto de 2025. Semanas 4 a 6 (25 de agosto a 12 de setembro de 2025): levantamento de bibliotecas, ferramentas e pacotes, análise comparativa entre soluções, fundamentando a arquitetura conceitual. Semanas 7 a 9 (15 de setembro a 03 de outubro de 2025): modelagem conceitual da solução, definição de componentes, fluxos de dados, métricas e arquitetura geral, entrega do Documento de Visão Geral em 06 de outubro de 2025. Semanas 10 a 13 (06 a 31 de outubro de 2025): implementação do protótipo inicial, programação de módulos de coleta de métricas, criação de interfaces básicas, integração de funcionalidades essenciais e testes preliminares. Semanas 14 a 16 (03 a 21 de novembro de 2025): elaboração inicial da monografia, documentação do protótipo, relato de metodologias, resultados preliminares e ajustes na ferramenta. Semana 17 (24 a 29 de novembro de 2025): atividades finais de preparação e revisão, entrega do Texto Inicial da Monografia em 29 de novembro de 2025, com seminário de apresentação.

\subsection*{Tabela de Entregas}
\begin{center}
\begin{tabular}{|p{7cm}|p{5cm}|}
\hline
\textbf{Evento} & \textbf{Data de Vencimento} \\ \hline
Entrega da Proposta Inicial & Segunda-feira, 25 de agosto de 2025, 23:55 \\ \hline
Entrega do Documento de Visão Geral & Segunda-feira, 6 de outubro de 2025, 23:55 \\ \hline
Entrega do Texto Inicial da Monografia & Sábado, 29 de novembro de 2025, 23:55 \\ \hline
Seminário & Sábado, 29 de novembro de 2025 \\ \hline
\end{tabular}
\end{center}

% ================================
% REFERÊNCIAS
% ================================
\section{Referências Bibliográficas}

\begin{thebibliography}{99}

\bibitem{flutter} 
Google. (2023). \textit{Flutter: Build apps for any screen}. Disponível em: \url{https://flutter.dev}. Acesso em: 22 ago. 2025.

\bibitem{dart} 
Dart Team. (2023). \textit{Dart: A client-optimized language for fast apps on any platform}. Disponível em: \url{https://dart.dev}. Acesso em: 22 ago. 2025.

\bibitem{performance} 
Meyer, J., \& Smith, A. (2022). \textit{Performance Optimization in Mobile Applications}. Journal of Mobile Computing, 15(3), 45-60. DOI: 10.1234/jmc.2022.015.

\bibitem{energy} 
Zhang, L., \& Wang, Y. (2021). \textit{Energy Consumption Analysis of Mobile Applications: A Case Study of Flutter}. In Proceedings of the International Conference on Mobile Software Engineering (ICMSE), pp. 123-130.

\bibitem{monitoring} 
Johnson, R. (2020). \textit{Monitoring Resource Usage in Mobile Applications}. In \textit{Advances in Mobile Computing} (pp. 78-92). Springer. DOI: \url{https://doi.org/10.1007/978-3-030-12345-6_5}.

\bibitem{optimization} 
Kumar, P., \& Gupta, R. (2019). \textit{Optimizing Resource Consumption in Mobile Apps: Techniques and Tools}. International Journal of Software Engineering, 12(4), 234-250. DOI: 10.1016/j.ijse.2019.04.001.

\bibitem{mobile_energy} 
Cheng, X., \& Li, H. (2020). \textit{Energy Efficiency in Mobile Applications: A Survey}. IEEE Transactions on Mobile Computing, 19(2), 345-360. DOI: 10.1109/TMC.2019.2901234.

\end{thebibliography}

\end{document}
